\chapter{Kooperationsvereinbarung}

zwischen 

1. Name des Kooperationspartners/Unternehmens vertreten durch Name der Vertreterin/des Vertreters – in der Folge "`die Projektpartnerin/der Projektpartner"' 

und 

2. Name der Schüler/innen – in der Folge "`das Projektteam"'

\centeredsection{Präambel}

Das Projektteam und die Projektpartnerin/der Projektpartner beabsichtigen gemäß der Verordnung über die abschließenden Prüfungen in den berufsbildenden mittleren und höheren Schulen, BGB. II Nr. 70/2000 vom 24.2.2000, die Planung und Durchführung eines Diplomprojekts, welches die Erstellung eines Konzeptes einer kostenoptimierten Instandhaltung als Ziel hat. Durch die Zusammenarbeit soll insbesondere den Mitgliedern des Projektteams die Möglichkeit eingeräumt werden, im Rahmen ihrer schulischen Ausbildung bei der Erstellung der Diplomarbeit an die Verhältnisse im technischen Berufsleben herangeführt zu werden, um dabei die in der Schule erworbenen theoretischen Kenntnisse und Fähigkeiten in der Praxis anzuwenden bzw. zu erweitern. Hingewiesen wird in diesem Zusammenhang auf den unentgeltlichen Charakter dieser Vereinbarung. 

\centeredsection{§ 1 Gegenstand}
Gegenstand ist die Erstellung von Arbeitsergebnissen zum Thema der Diplomarbeit. Dieses Thema ist der Projektbeschreibung und dem Pflichtenheft zu entnehmen, welches der Kooperationsvereinbarung beiliegt. Die Projektpartnerin/Der Projektpartner wird jedoch darauf hingewiesen, dass es sich um ein Projekt im Zusammenhang mit der schulischen Ausbildung handelt und daher jede Haftung des Projektteams, insbesondere in Hinsicht auf die Unentgeltlichkeit des Vertrages, ausgeschlossen ist. Nutzungs- und Verwertungsrechte von im Rahmen dieser Vereinbarung erstellten Arbeitsergebnissen stehen der Projektpartnerin/dem Projektpartner sowie dem Projektteam gemeinsam zu. 

\centeredsection{§ 2 Laufzeit}
Die vorliegende Kooperation tritt am …………… in Kraft und wird bis zum Ende der Reife- und Diplomprüfung der HTL Pinkafeld abgeschlossen. 

\centeredsection{§ 3 Rechte und Pflichten des Projektteams}
Die Mitglieder des Projektteams haben das Recht, die Räumlichkeiten der Projektpartnerin/des Projektpartners samt Infrastruktur und EDV-Infrastruktur im für die Projektabwicklung erforderlichen Ausmaß nach vorheriger schriftlicher Genehmigung durch die Projektpartnerin/den Projektpartner mitzubenutzen. Das Projektteam verpflichtet sich, die im Gegenstand genannten Arbeiten sorgfältig und unter möglichster Schonung der Interessen der Projektpartnerin/des Projektpartners durchzuführen. Das Projektteam unterliegt der Betriebsordnung der Projektpartnerin/des Projektpartners. Das Projektteam verpflichtet sich zur Geheimhaltung aller ihm zur Kenntnis gelangenden Geschäfts- und Betriebsgeheimnisse. 

\centeredsection{§ 4 Rechte und Pflichten der Projektpartnerin/des Projektpartners}
Die Projektpartnerin/Der Projektpartner verpflichtet sich, dem Projektteam beratend zur Verfügung zu stehen und alles zu unterlassen, was der Vollendung des Projekts entgegensteht. Die Projektpartnerin/Der Projektpartner verpflichtet sich, dem Projektteam folgende Hilfsmittel zur Verfügung zu stellen: 
\begin{itemize}
    \item ...
    \item ...
\end{itemize}

Sollte das Projektteam im Rahmen dieser Kooperationsvereinbarung eine Erfindung machen, die nach dem Gebrauchsmustergesetz bzw. dem Patentgesetz (PatG) schützbar ist, gilt diese Erfindung als Diensterfindung im Sinne des PatG und die §§ 6-19 PatG (in der geltenden Fassung) entsprechend. Das Projektteam verpflichtet sich, die Projektpartnerin/den Projektpartner von einer im Rahmen der Kooperationsvereinbarung gemachten Erfindung unverzüglich in Kenntnis zu setzen. Die Projektpartnerin/Der Projektpartner hat daraufhin das Recht, binnen vier Wochen ab dieser Bekanntgabe zu erklären, dass sie/er das Patentrecht für dich beansprucht. In diesem Fall steht dem Projektteam eine entsprechende Vergütung nach den einschlägigen Bestimmungen des PatG (in der geltenden Fassung) zu. Sollte das Projektteam im Rahmen dieser Kooperationsvereinbarung ein Werk schaffen, dem Schutz im Sinne des Urheberrechtsgesetzes zukommt, verpflichtet es sich, die Projektpartnerin/den Projektpartner davon unverzüglich zu informieren. Die Projektpartnerin/Der Projektpartner hat daraufhin die Möglichkeit, binnen vier Wochen ab dieser Bekanntgabe, mit dem Projektteam einen Werknutzungsvertrag abzuschließen. 

\centeredsection{§ 5 Einsicht und Präsentation}
Da die Tätigkeit des Projektteams auch Inhalt bzw. Grundlage der an der HTL Pinkafeld zu erstellenden Diplomarbeit ist, berechtigt die Projektpartnerin/der Projektpartner die zuständigen Organe des Bundes zur Einsicht und Kontrolle, um die in der Verordnung über die abschließenden Prüfungen an den berufsbildenden höheren Schulen genannten Aufgaben zu erfüllen. Das Projektteam ist auch berechtigt, Ergebnisse der Diplomarbeit bei der mündlichen Reife- und Diplomprüfung zu präsentieren. Die zuständigen Organe des Bundes sind ihrerseits wiederum gegenüber jedermann zur Geschäfts- und Betriebsgeheimnisse der Projektpartnerin/des Projektpartners verpflichtet. 

\vspace*{1cm}

\hspace*{\fill}\begin{tabular}{cp{2em}c} 
   \hspace{4cm}        & & \hspace{6cm} \\\cline{1-1}\cline{3-3}
                       & & \\[-3mm]
   {\footnotesize Ort, Datum }  & & {\footnotesize Projektpartner/in }
\end{tabular}\hspace*{\fill}

\vspace*{1cm}

\hspace*{\fill}\begin{tabular}{cp{2em}c} 
   \hspace{4cm}        & & \hspace{6cm} \\\cline{1-1}\cline{3-3}
                       & & \\[-3mm]
   {\footnotesize Ort, Datum }  & & {\footnotesize Projektteam }
\end{tabular}\hspace*{\fill}
