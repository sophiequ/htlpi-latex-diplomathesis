\chapter{Projektvorstudie}

\begin{itemize}
\item Istsituation (Description of actual condition, Use cases, activity diagrams)
\item Stärken-, Schwächen Analyse (Strength-, Weaknesses analysis), evtl. SWOT
\item Generelle Ziele und (davon abgeleitete) konkrete (messbare) Zielsetzung und Ergebnisse (Goals, objectives and deliverables)
\item Grobe Anforderungen (Requirements on a higher level)
\item Alternative Lösungen (Alternative solutions)
\item Entscheidungsfindung (Decision for a solution, Value-benefit analysis)
\item Evtl Machbarkeitsstudie (feasibility study)
\item Grobschätzung (Rough estimation, Work breakdown structure (WBS))
\item Kosten-Nutzen Analyse (Cost-benefit analysis)
\item Achtung: Wenn Teile der Vorstudie einen speziellen Bezug zu einer der individuellen Themenstellung eines der Diplomanden haben, dann sind diese in dessen individuellen Teil anzuführen. (Beispiel: Schüler A hat die Themenstellung “...unter Berücksichtigung der Barrierefreiheit”, dann sind diese Analysen in dessen Teil anzuführen!)

\end{itemize}
