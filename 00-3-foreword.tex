\chapter*{Vorwort}
\addcontentsline{toc}{chapter}{Vorwort}

persönlicher Zugang zur Arbeit, d.h. jedes Mitglied beschreibt seinen Zugang zur Themenstellung, am Beginn zumindest ein Einleitungsabsatz, Zeit: Gegenwart

\rule{\linewidth}{0.5pt}

\section*{Beispiele für mögliche Kapitel}
Wichtig ist, dass es einen einleitenden gemeinsamen Teil gibt und danach die zusammenhängenden individuellen Teile, der jeweiligen Kandidaten.
Der untenstehende Vorschlag ist lediglich ein Anhaltspunkt, welche Kapitel enthalten sein können. Letztlich wird der Inhalt auch sehr stark von der Art der Aufgabenstellung abhängen. Z.B. ob und in welcher Form eine Vorstudie zu machen ist, wird davon abhängen, was am Anfang eines Projektes analysiert werden soll. Gerade bei diesen Aspekten ist es wichtig sie mit dem Projektbetreuer abzusprechen. 
Grundsätzlich gilt, dass man für die nächsten Arbeitsschritte einen Vorschlag erstellt und diesem mit dem Projektbetreuer bespricht.
